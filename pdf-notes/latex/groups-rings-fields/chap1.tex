

\section{Groups}

\begin{framed}
\begin{definition}[Group]
    A group is a set $G$ together with an element $e\in G$, the neutral element, as well as an operation ("Verknüpfung") $G\cdot G \rightarrow G$ that satisfies
    \begin{enumerate}[itemsep=3pt, topsep=3pt]
        \item $a\cdot (b\cdot c) = (a\cdot b) \cdot c $ ... Associativity 
        \item $e\cdot g = g$ ... neutral element
        \item $g' \cdot g = e$ ... inverse element
    \end{enumerate}
    Its notated by the triple: $(G,e,\cdot)$
\end{definition}
\end{framed}
Th notation of the operation means, that $\cdot$ takes two elements and ouputs a third element, all from $G$.  
An example of a group is $(\mathbb{R}, 0, +)$. The triple $(\mathbb{N},0,+)$ is not a group, as its members dont have inverse elements.

\begin{framed}
\begin{lemma}[Group properties]
    \label{lem:groupproperties}
    \begin{enumerate}[itemsep=3pt, topsep=3pt]
        \item the neutral elemenet of a group is unique
        \item the inverse element of $g$ is unique, which allows to write $g^{-1}$
        \item for all $a,b\in G$ we have $(a^{-1})^{-1} = a$ and $(ab)^{-1} = a^{-1} b^{-1}$
        \item for all $a,b,c\in G$ we have $ab=ac$ if and only if $b=c$. \\Same for $ba=ca$
    \end{enumerate}
\end{lemma}
\end{framed}
\begin{proof}
    TODO
\end{proof}

\begin{framed}
\begin{definition}[Abelian group]
    A group is called abelian ("abelsch") if it is commuative: $a\cdot b = b\cdot a$
\end{definition}
\end{framed}

\begin{framed}
\begin{definition}[Subgroup]
    A subset of $G$ is a subgroup of $G$ if it is a group.
\end{definition}
\end{framed}

\begin{framed}
\begin{definition}[Homomorphism]
Let $(G,\cdot)$ and $(H,*)$ be groups. A mapping $\phi : G\rightarrow H$ is a homomorphism if
$$\phi(a \cdot b) = \phi(a) * \phi(b) $$
for all $a,b\in H$
\end{definition}
\end{framed}

\begin{framed}
\begin{definition}[Isomorphism]
    A bijective homomorphism is an isomorphism.
\end{definition}
\end{framed}

\begin{framed}
\begin{lemma}[Properties of homomorphisms]
    Let $\phi$ be a homomorphism and $e_i$ be the neutral element of group $i$.
    \begin{enumerate}[itemsep=3pt, topsep=3pt]
        \item $\phi(e_G) = e_H$
        \item $\phi(a^{-1}) = \phi(a)^{-1}$
    \end{enumerate}
\end{lemma}
\end{framed}
\begin{proof}
    $\text{\\}$
    \begin{enumerate}[itemsep=3pt, topsep=3pt] 
        \item by above definitions: $$\phi(e_G) = \phi(e_G\cdot e_G) = \phi (e_G)* \phi(e_G)$$
        Now apply $\phi(e_G)^{-1}$ from left
        $$e_H = \phi(e_G)^{-1} * \phi (e_G)* \phi(e_G) = e_H *\phi (e_G) = \phi(e_G) $$
        \item Let $a\in G$. We just showed that
        $$ \phi(a) * \phi(a^{-1}) = \phi (a\cdot a^{-1}) = \phi (e_G) = e_H$$
        But we also know that
        $$  \phi(a^{-1}) * \phi(a) = \phi (a^{-1} \cdot a) = \phi (e_G) = e_H$$
        As the inverse element is unique (see lemma \ref{lem:groupproperties}), the statement follows.
    \end{enumerate}
\end{proof}



\section{Rings and fields}
(German: "Ringe" und "Körper")

\begin{framed}
\begin{definition}[Ring] A ring is a set $R$ with the two operations addition and multiplication:
    \begin{align}
        +     &: R\times R \rightarrow R \\
        \cdot &: R\times R \rightarrow R
    \end{align}
    and the following properties:
    \begin{itemize}[itemsep=3pt, topsep=3pt]
        \item $R$ is an ablian Group
        \item $\cdot$ is associative
        \item it holds that $a\cdot(b+c) = a\cdot b + a\cdot c$
    \end{itemize}
\end{definition}
\end{framed}

A ring is called unitary ring or ring with unity if $\exists 1 \in R$ st. $1\cdot a = a\cdot 1 = a \forall a\in R$ this element is called unity- or one-element.\\

Apparently now one can already proof fun statements like this:
\begin{framed}
\begin{lemma}[Good to know lemma]
    Seemingly
    $$a \cdot 0 = 0$$
\end{lemma}
\end{framed}
\begin{proof}
    Take $0+ 0 = 0$  and distributivity:
    $$0\cdot a = (0+0)\cdot a = 0\cdot a+ 0\cdot a$$
    add the inverse of (fancy way of saying subtract) $0\cdot a$ and use associativity
    $$0 = 0\cdot a - 0\cdot a (0\cdot a + 0\cdot a) - 0\cdot a = 0\cdot a + (0\cdot a -0\cdot a)= 0\cdot a + 0 = 0\cdot a$$
    Beautiful.
\end{proof}

\begin{framed}
\begin{definition}[Definition of fields based on groups] ("Körper")
    A field is a unitary ring where each nonzero element has a multiplicative invers. More explicit:\\
    A tuple $(K, +, \cdot ,0,1)$ where 
    \begin{align}
        +     &: K\times K \rightarrow K \\
        \cdot &: K\times K \rightarrow K
    \end{align}
    and where $K$ is a set, is called field if 
    \begin{itemize}[itemsep=3pt, topsep=3pt]
        \item $K$ together with addition is an abelian group with neutral element $0$
        \item $K \backslash \{0\}$ together with multiplication is an abelian group with neutral element $1$
        \item distributivity: $a\cdot (b+c) = a\cdot b + a\cdot c$
    \end{itemize}
\end{definition}
\end{framed}

\begin{framed}
\begin{definition}[Axiomatic definition of fields]
    A tuple $(K, +, \cdot ,0,1)$ where 
    \begin{align}
        +     &: K\times K \rightarrow K \\
        \cdot &: K\times K \rightarrow K
    \end{align}
    and where $K$ is a set, is called field if the following axioms hold
    \begin{itemize}[itemsep=3pt, topsep=3pt]
        \item associativity, commutativity, existence of neutral element, and existence of inverse element of addition
        \item associativity, commutativity, existence of neutral element, and existence of inverse element of multiplication
        \item distributivity of addition and multiplication
        \item $1\neq 0$
    \end{itemize}
\end{definition}
\end{framed}

\begin{framed}
\begin{lemma}[Properties of fields] We have that:
    \begin{itemize}[itemsep=3pt, topsep=3pt]
        \item Every field has at least two elements
        \item $0\cdot a = 0$
        \item Fields dont have zero divisors, in other words: $a\cdot b= 0 \implies a=0 \vee b=0$
        \item $a\cdot (-b) = -(a\cdot b)$ and $(-a)\cdot (-b) = a\cdot b$
        \item $x\cdot a = y\cdot a $ with $a\neq 0 \implies x=y$
    \end{itemize}
\end{lemma}
\end{framed}
\begin{proof}
    TODO
\end{proof}